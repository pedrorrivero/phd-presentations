%%    _____  _____
%%   |  __ \|  __ \    AUTHOR: Pedro Rivero
%%   | |__) | |__) |   ---------------------------------
%%   |  ___/|  _  /    DATE: November 10, 2021
%%   | |    | | \ \    ---------------------------------
%%   |_|    |_|  \_\   https://github.com/pedrorrivero
%%

\section{Conclusions and future research}

%% ----------------------------------------------------------------------------
%% ----------------------------------------------------------------------------

\begin{frame}{Conlcusions}

	\begin{itemize}
		\item I showcased an instance of hadron mass generation in QCD using the NJL model in $1+1$ dimensions and $2$ flavors.
    \item I discovered a clear transition from a regime dominated by the quark masses, to a regime dominated by their interaction.
    \item I compiled and developed the computational techniques necessary for efficiently simulating Quantum Field Theories on a quantum computer.
    \item I established how revealing these kind of problems can be when addressed on quantum computers, and why Quantum Field Theory should remain a key motivation for developing QIS methods and techniques.
	\end{itemize}

\end{frame}

%% ----------------------------------------------------------------------------

\begin{frame}{Future research}

  \begin{itemize}
    \item Calculation of parton distribution functions and form factors for the study of Deep inelastic scattering.
    \item Repeat calculations with other mappings and parametrization ansatzes.
    \item Increase the dimensions of our problem to $2+1$, and $3+1$.
    \item Explore other quantum field theories up to full QCD.
    \item Formalize the scalabilty of these techniques in the NISQ era and beyond.
    \item Develop a robust notation for these applications of QIS.
  \end{itemize}

\end{frame}
